\documentclass[10pt]{letter}
\usepackage{UPS_letterhead,xcolor,mhchem,mathpazo,ragged2e,hyperref}
\newcommand{\alert}[1]{\textcolor{red}{#1}}
\definecolor{darkgreen}{HTML}{009900}


\begin{document}

\begin{letter}%
{To the Editors of the Journal of Chemical Physics}

\opening{Dear Editors,}

\justifying
Please find attached a revised version of our \textit{Communication} entitled 
\begin{quote}
\textit{``How accurate are EOM-CC4 vertical excitation energies?''}.
\end{quote}
We thank the reviewers for their constructive comments.
Our detailed responses to their comments can be found below.
For convenience, changes are highlighted in red in the revised version of the manuscript. 

We look forward to hearing from you.

\closing{Sincerely, the authors.}

%%% REVIEWER 1 %%%
\noindent \textbf{\large Authors' answer to Reviewer \#1}

\begin{itemize}

	\item 
	{In this short manuscript, the CC4 method is applied to excitation energies at the first time. 
	The manuscript is well-written, and the good performance of the CC4 approach for excitation energies is an important piece of information for computational chemists who are interested in calculating benchmark-quality excitation energies. 
	Nevertheless, the paper would probably benefit from the extension of the study. 
	E.g., the performance of the CCn methods is relatively weak for potential energy surfaces or geometries. 
	It would be instructive to see how CC4 performs for excited states.}
	\\
	\alert{We thank the reviewer for his/her positive comments.
	The present \textit{Communication} deals specifically with vertical transition energies.
	Although extension to excited-state potential energy surfaces would definitely deliver some interesting new insights on the performance of CC4, we believe that it is outside the scope of the present study.
	However, we hope to answer these important questions in a future manuscript.}

	\item 
	{In the abstract and conclusions, the approximate error of the approach should be given in eV instead of kJ/mol as excitation energies are usually presented in eV.}
	\\
	\alert{We have added the corresponding value in eV.}
		
	\item 
	{The terms ``singly/doubly-excited states'' or ``single/double excitations'' are misleading. 
	E.g., ``states dominated by single/double excitations'' should be used instead. }
	\\
	\alert{We have modified these terms in accordance with the reviewer's comment.}
	
	\item 
	{Tables: the numbers in parentheses at the FCI values is probably the uncertainty of the FCI excitation energy. 
	This should be mentioned in the caption of the tables. }
	\\
	\alert{Thank you for spotting this. This has been added.}
	
	\item 
	{Sect. II, last para: system considered $\to$ systems considered.}
	\\
	\alert{Fixed.}

	\item 
	{Ref. 69: gaussian-2 $\to$ Gaussian-2}
	\\
	\alert{Fixed.}
		
\end{itemize}

 
\end{letter}
\end{document}






