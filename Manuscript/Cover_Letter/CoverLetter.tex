\documentclass[10pt]{letter}
\usepackage{UPS_letterhead,xcolor,mhchem,mathpazo,ragged2e}
\newcommand{\alert}[1]{\textcolor{red}{#1}}
\definecolor{darkgreen}{HTML}{009900}


\begin{document}

\begin{letter}%
{To the Editors of the Journal of Chemical Physics}

\opening{Dear Editors,}

\justifying
Please find enclosed our manuscript entitled \textit{``How accurate are EOM-CC4 vertical excitation energies?''}, which we would like you to consider as a \textit{Communication} in the \textit{Journal of Chemical Physics}.

In this communication, we report what we believe is the very first investigation of the performance of EOM-CC4, an approximate equation-of-motion coupled-cluster model which includes iterative quadruple excitations, for vertical excitation energies in molecular systems.
For a set of 28 excited states in 10 small molecules, CC4 is compared to other coupled-cluster models up EOM-CCSDTQP as well as FCI.
The present results nicely illustrate the systematic improvement of the transition energies when one ramps up the computational effort following the series CC2, CCSD, CC3, CCSDT, CC4, and CCSDTQ.
Our main conclusions are that CC4 is a rather competitive approximation to its more expensive CCSDTQ parent as well as a very significant improvement over both its third-order version, CC3, and the ``complete'' CCSDT method.
This is particularly true in the case of single excitations when one reaches sub-kJ~mol$^{-1}$ accuracy at the CC4 level.
Our results also evidence that, although the same qualitative conclusions hold, one cannot reach the same level of accuracy for transitions of double excitation nature.

Due to the usefulness of benchmark sets and their corresponding reference data for the electronic structure community, we expect this work to be of interest to a wide audience within the chemistry and physics communities.
We suggest Jurgen Gauss, John Stanton, So Hirata, Marcel Nooijen, Henrik Koch, and Don Truhlar as potential referees.	
We look forward to hearing from you soon.

\closing{Sincerely, the authors.}


\end{letter}
\end{document}






