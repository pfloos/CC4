\documentclass[aip,jcp,reprint,noshowkeys,superscriptaddress]{revtex4-1}
\usepackage{graphicx,dcolumn,bm,xcolor,microtype,multirow,amscd,amsmath,amssymb,amsfonts,physics,wrapfig,txfonts}
\usepackage[version=4]{mhchem}

\usepackage{siunitx}
\DeclareSIUnit[number-unit-product = {\,}]
\cal{cal}
\DeclareSIUnit\kcal{\kilo\cal}

\newcommand{\alert}[1]{\textcolor{black}{#1}}
\usepackage[normalem]{ulem}
\newcommand{\titou}[1]{\textcolor{red}{#1}}
\newcommand{\trashPFL}[1]{\textcolor{red}{\sout{#1}}}
\newcommand{\PFL}[1]{\titou{(\underline{\bf PFL}: #1)}}

\newcommand{\ie}{\textit{i.e.}}
\newcommand{\eg}{\textit{e.g.}}
\newcommand{\mc}{\multicolumn}
\newcommand{\fnm}{\footnotemark}
\newcommand{\fnt}{\footnotetext}
\newcommand{\tabc}[1]{\multicolumn{1}{c}{#1}}

% softwares
\newcommand{\QP}{\textsc{quantum package}}
\newcommand{\MRCC}{\textsc{mrcc}}
\newcommand{\CFOUR}{\textsc{cfour}}
\newcommand{\DALTON}{\textsc{dalton}}


% variables
\newcommand{\hH}{\Hat{H}}
\newcommand{\hT}{\Hat{T}}
\newcommand{\bH}{\Bar{H}}

% excited states
\newcommand{\npis}{n \to \pi^*}
\newcommand{\pipis}{\pi \to \pi^*}

\usepackage[
	colorlinks=true,
    citecolor=blue,
    breaklinks=true
	]{hyperref}
\urlstyle{same}

\begin{document}

\newcommand{\LCPQ}{Laboratoire de Chimie et Physique Quantiques (UMR 5626), Universit\'e de Toulouse, CNRS, UPS, France}
\newcommand{\SMU}{Department of Chemistry, Southern Methodist University, Dallas, Texas 75275, USA}
\newcommand{\UOP}{Dipartimento di Chimica e Chimica Industriale, University of Pisa, Via Moruzzi 3, 56124 Pisa, Italy}
\newcommand{\CEISAM}{Universit\'e de Nantes, CNRS,  CEISAM UMR 6230, F-44000 Nantes, France}

\title{How accurate is CC4 for vertical excitation energies?}

\author{Pierre-Fran\c{c}ois Loos}
	\email{loos@irsamc.ups-tlse.fr}
	\affiliation{\LCPQ}
\author{Devin A.~Matthews}
	\affiliation{\SMU}
\author{Filippo Lipparini}
	\affiliation{\UOP}
\author{Denis Jacquemin}
	\email{denis.jacquemin@univ-nantes.fr}
	\affiliation{\CEISAM}

% Abstract
\begin{abstract}
\titou{Here comes the abstract.}
\end{abstract}

% Title
\maketitle

%%%%%%%%%%%%%%%%%%%%%%%%%%%%%%%%
\section{Introduction}
%%%%%%%%%%%%%%%%%%%%%%%%%%%%%%%%

Single-reference coupled-cluster (CC) theory provides a hierarchy of size-extensive methods which delivers increasingly accurate energies and properties via the systematic increase of the maximum excitation degree of the cluster operator $\hT = \hT_1 + \hT_2 + \ldots + \hT_k$ (where $k \ge n$ and $n$ is the number of electrons).
Without any truncation (\ie, $k = n$), the so-called full CC (FCC) method is equivalent to full configuration interaction (FCI), hence providing the exact energy and wave function of the system within a given basis set.
However, it is not computationally viable due to its exponential scaling with system size, and one has to resort to truncated CC methods  (\ie, $k \ll n$) for computational convenience.
Popular choices are CC with singles and doubles (CCSD), CC with singles, doubles, and triples (CCSDT), CC with singles, doubles, triples, and quadruples (CCSDTQ), and 
CC with singles, doubles, triples, quadruples, and pentuples (CCSDTQP) with corresponding computational scalings of $\order*{N^{6}}$, $\order*{N^{8}}$,  $\order*{N^{10}}$, and  $\order*{N^{12}}$, respectively (where $N$ denotes the number of orbitals).
An alternative, systematically-improvable family of methods is defined by the CC2, CC3, and CC4 series of models which have been introduced by the Aarhus group in the context of CC response theory.
These methods scale as $\order*{N^{5}}$, $\order*{N^{7}}$, and $\order*{N^{9}}$, respectively, and can be seen as cheaper approximations of CCSD, CCSDT, and CCSDTQ, by skipping the most expensive terms and avoiding the storage of the higher-excitation amplitudes.
A somehow similar strategy has been applied to produce the CCSDT-3 and CCSDTQ-3 models based on arguments stemming from perturbation theory.
Of course, many more approximate CC models have been developed and we refer the interested reader to specialized reviews for more details.

Coupled-cluster methods have been particularly successful for small- and medium-sized molecules in the field of thermodynamics, kinetics, and spectroscopy, thanks to the computations of accurate geometries, potential energy surfaces, couplings between electronic states, and a vast panel of properties such as dipoles (and higher moments), vibrational frequencies, NMR chemical shifts, magnetizabilities, polarizabilities, etc.
Although originally developed for ground-state energies and properties, CC has been successfully extended to excited states thanks to the equation-of-motion (EOM) and linear-response (LR) formalisms which are known to produce identical excitation energies but different properties.
In EOM-CC, one determines vertical excitation energies via the diagonalization of the similarity-transformed Hamiltonian $\bH = e^{-\hT} \hH e^{\hT}$.
A more general procedure to compute excitation energies that can be applied to any approximate CC model (as the ones mentioned above) consists in diagonalizing the so-called CC Jacobian obtained via the differentiation of the CC amplitude equations.
Again, by systematically increasing the maximum excitation degree of $\hT$, one can systematically produce increasingly accurate excitation energies with the ``complete'' series CCSD, CCSDT, CCSDTQ, CCSDTP, \ldots or the ``approximate'' series CC2, CC3, CC4, \ldots while retaining the formal scaling of their ground-state analog.

Recently, a large collection of deterministic, stochastic or hybrid selected CI (SCI) methods has (re)appeared in the electronic structure landscape providing an alternative route to highly-accurate ground- and excited-state energies.
The general idea behind SCI methods is simple: rather than exploring the entire FCI space by systematically increasing the maximum excitation degree of the determinants taken into account (leading to the slowly convergent and size-inconsistent series of methods CISD, CISDT, CISDTQ, \ldots), one performs a sparse exploration of the FCI space by selecting only the most energetically-important determinants thanks to a suitable criterion usually based on perturbation theory.
By iteratively increasing the number of determinants of the variational space and supplemented it with a second-order perturbative correction (PT2), the SCI+PT2 family of methods has been recently shown to produce near-FCI correlation and excitation energies for small- and medium-size molecules in compact basis sets.

Taking advantage of the high accuracy of CC and SCI+PT2 methods, we have very recently created a large dataset gathering more than 500 highly-accurate vertical excitation energies of various natures (valence, Rydberg, $\npis$, $\pipis$, singlet, doublet, triplet, charge-transfer, and double excitations) in small- and medium-sized molecules ranging from diatomics to molecules as large as naphthalene. 
The main purpose of this so-called QUEST database is to provide reference excitation energies in order to perform fair and reliable comparisons between electronic structure methods and assess their strengths and weakness in a large panel of chemical scenarios.
Most of these reference transition energies, which rely exclusively on high-level \textit{ab initio} calculations, can be reasonably considered as chemically accurate, \ie, within \SI{1}{\kcal\per\mol} or \SI{0.043}{\eV} of the FCI limit.
However, their accuracy may rapidly deteriorate, in particular, as the system size grows.
Indeed, it is usually challenging to compute reliable SCI+PT2 estimates or CCSDTQ excitation energies for molecules with more than 4 non-hydrogen atoms.
Therefore, for the larger molecules of the QUEST database, we mostly relied on CCSDT to define reference excitation energies.
Of course, it would be highly valuable to have access to more accurate methods (including, at least, quadruple excitations) in order to refine, in the near future, these theoretical best estimates.

In this context, the main purpose of the present study is to assess the relative accuracy of the approximate CC4 model against the most expensive CCSDTQ and CCSDTQP methods in the case of vertical excitation energies.
We are also interested in its absolute accuracy with respect to FCI.
To do so, we consider a set of 8 small molecules (\ce{NH3}, \ce{BH}, \ce{BF}, \ce{CO}, \ce{N2}, \ce{HCl}, \ce{H2S}, and \ce{H2O}) and we compare the excitation energies associated with 25 singlet excited states of various natures ($\npis$, $\pipis$, Rydberg, valence, charge-transfer, \ldots) and spatial symmetries obtained with various high-level CC methods.
Although several studies have been published on the performance of CC4 for ground-state energies and properties, to the best of our knowledge, CC4 has never been considered to compute excited-states energies before.
As we shall see below, CC4 is an excellent approximation to its CCSDTQ parent, and produces excitation energies with sub-\si{\kJ\per\mol} accuracy for this set of small molecules, well below the chemical accuracy threshold.

%%%%%%%%%%%%%%%%%%%%%%%%%%%%%%%%
\section{Computational details}
%%%%%%%%%%%%%%%%%%%%%%%%%%%%%%%%

All the methods considered in the present study are listed in Table \ref{tab:scaling} alongside their formal computational scaling and the electronic structure software employed to compute the excitation energies.
In a nutshell, we have used by default the {\CFOUR} program to compute the CC energies at the notable exception of CCSDTQP where we have employed {\MRCC} and its automated implementation of high-order CC methods based on the strings of spin-orbital indices.
The FCI estimates were obtained with the SCI algorithm known as \textit{``configuration interaction using a perturbative selection made iteratively''} (CIPSI) implemented in {\QP}.
The error bars associated with these CIPSI calculations have been computed using our recently-developed protocol presented in Ref.~\onlinecite{Veril_2021}.
The present {\CFOUR} calculations have been performed with the new and fast CC module (\texttt{xncc}) written by one of author (DAM) which couples a general algebraic
and graphical interpretation of the non-orthogonal spin-adaptation approach with highly efficient storage format and set of implementation techniques designed to minimize data movement and to avoid costly tensor transposes.

All calculations have been performed in the frozen-core calculations and the geometries of the system considered here have been extracted from previous studies at the exception of \ce{BH} and \ce{BF}.
\titou{Two basis sets are used (aug-cc-pVDZ and aug-cc-pVTZ). 
For the small ones, we could compute CCSDTQP energies but not in the larger one.}

%%%%%%%%%%%%%%%%%%%%%%%%%%%%%%%%
\section{Results and discussion}
%%%%%%%%%%%%%%%%%%%%%%%%%%%%%%%%

\begin{table}
	\caption{Methods considered, their formal computational scaling and the name of the electronic structure software where it has been implemented.
	Here, $N$ is the number of orbitals.
	\label{tab:scaling}}
	\begin{ruledtabular}
	\begin{tabular}{lccc}
		Method	&	Scaling					&	Code		&	Ref.					\\
		\hline
		CC3			&	$\order*{N^{7}}$	&	\DALTON		&	\onlinecite{dalton}		\\
		CCSDT-3		&	$\order*{N^{7}}$	&	\CFOUR		&	\onlinecite{cfour}		\\
		CCSDT		&	$\order*{N^{8}}$	&	\CFOUR		&	\onlinecite{cfour}		\\
		CC4			&	$\order*{N^{9}}$	&	\CFOUR		&	\onlinecite{cfour}		\\
		CCSDTQ-3	&	$\order*{N^{9}}$	&	\CFOUR		&	\onlinecite{cfour}		\\
		CCSDTQ		&	$\order*{N^{10}}$	&	\CFOUR		&	\onlinecite{cfour}		\\
		CCSDTQP		&	$\order*{N^{12}}$	&	\MRCC		&	\onlinecite{mrcc}		\\	
		CIPSI		&	$\order*{e^{N}}$	&	\QP			&	\onlinecite{qp2}		\\
	\end{tabular}
	\end{ruledtabular}
\end{table}

\begin{squeezetable}
\begin{table*}
	\caption{Vertical excitation energies (in eV) of a selection of molecular excited states obtained at various levels of theory with the aug-cc-pVDZ and aug-cc-pVTZ basis sets.
	\label{tab:BigTab}}
	\begin{ruledtabular}
	\begin{tabular}{llrrrrrrrrrrrrrr}
				&		&	\mc{8}{c}{aug-cc-pVDZ}		&		\mc{6}{c}{aug-cc-pVTZ}		\\	
				\cline{3-10} \cline{11-16}
	Mol.	&	State				&CCSD	&CC3	&CCSDT-3&CCSDT	&CC4	&CCSDTQ	&CCSDTQP	&FCI	
									&CCSD	&CC3	&CCSDT	&CC4	&CCSDTQ	&FCI			\\
	\hline
	\ce{NH3}	&	$^1A_2$ 		&6.455	&6.464	&6.472	&6.462	&6.479	&6.480	&6.482	&6.483(1)	&6.600	&6.573	&6.571	&6.585	&6.586	&6.593(22)	\\	
				&	$^1E$			&8.024	&8.061	&8.065	&8.057	&8.078	&8.079	&8.081	&8.082(1)	&8.148	&8.146	&8.143	&8.161	&8.161	&8.171(20)	\\	
				&	$^1A_1$ 		&9.649	&9.664	&9.668	&9.659	&9.677	&9.677	&9.680	&9.681(8)	&9.334	&9.318	&9.314	&9.331	&9.331	&9.340(19)	\\
				&	$^1A_2$ 		&10.376	&10.396	&10.402	&10.391	&10.409	&10.409	&10.411	&10.412(1)	&9.953	&9.945	&9.939	&9.957	&9.957	&9.967(19)	\\
	\ce{BH}		&	$^1\Pi$ 		&2.970	&2.955	&2.958	&2.946	&2.947	&2.947	&2.947	&2.947(0)	&2.928	&2.910	&2.900	&2.901	&2.901	&2.901(0)	\\
	\ce{BF}		&	$^1\Pi$ 		&6.534	&6.478	&6.491	&6.491	&6.484	&6.485	&6.485	&6.485(1)	&6.464	&6.410	&6.423	&6.416	&6.417	&6.418(2)\\
	\ce{CO}		&	$^1\Pi$ 		&8.671	&8.572	&8.591	&8.574	&8.562	&8.563	&8.561	&8.565(2)	&8.587	&8.486	&8.492	&8.479	&8.480	&	\\
				&	$^1\Sigma^-$ 	&10.096	&10.122	&10.112	&10.062	&10.055	&10.057	&10.057	&10.056(1)	&9.986	&9.992	&9.940	&9.930	&9.932	&	\\
				&	$^1\Delta$ 		&10.210	&10.225	&10.220	&10.178	&10.167	&10.169	&10.168	&10.168(1)	&10.123	&10.119	&10.076	&10.064	&10.066	&	\\
				&	$^1\Sigma^+$ 	&11.171	&10.917	&10.989	&10.944	&10.925	&10.926	&10.919	&			&11.222	&10.943	&10.987	&10.961	&10.963	&	\\
				&	$^1\Sigma^+$ 	&11.710	&11.483	&11.543	&11.518	&11.510	&11.510	&11.506	&			&11.751	&11.489	&11.540	&11.521	&11.523	&	\\
				&	$^1\Pi$ 		&11.973	&11.737	&11.802	&11.767	&11.757	&11.758	&11.753	&			&11.960	&11.690	&11.737	&11.719	&11.720	&	\\
	\ce{N2}		&	$^1\Pi_g$  		&9.495	&9.442	&9.450	&9.417	&9.409	&9.411	&9.409	&9.411(3)	&9.408	&9.344	&9.326	&9.317	&9.319	&	\\
				&	$^1\Sigma_u^-$	&10.197	&10.059	&10.063	&10.060	&10.063	&10.055	&10.054	&10.054(0)	&9.996	&9.885	&9.890	&9.883	&9.878	&9.879(4)\\
				&	$^1\Delta_u$ 	&10.607	&10.433	&10.449	&10.436	&10.439	&10.430	&10.428	&10.429(0)	&10.443	&10.293	&10.302	&10.294	&10.287	&10.289(12)\\
				&	$^1\Sigma_g^+$	&13.326	&13.229	&13.270	&13.202	&13.171	&13.182	&13.181	&13.180(1)	&13.151	&13.013	&12.999	&12.962	&12.974	&	\\
				&	$^1\Pi_u$  		&13.451	&13.279	&13.323	&13.174	&13.128	&13.131	&13.127	&			&13.422	&13.223	&13.140	&13.091	&13,095	&	\\
				&	$^1\Sigma_u^+$	&13.250	&13.146	&13.186	&13.130	&13.099	&13.109	&13.107	&			&13.263	&13.120	&13.118	&13.078	&13.090	&	\\
				&	$^1\Pi_u$ 		&13.765	&13.635	&13.663	&13.591	&13.551	&13.560	&13.558	&			&13.674	&13.494	&13.455	&13.409	&13.419\\
	\ce{HCl}	&	$^1\Pi$ 		&7.862	&7.819	&7.828	&7.815	&7.822	&7.822	&7.823	&7.823(0)	&7.906	&7.840	&7.834	&7.837	&7.837	&7.838(1)\\
	\ce{H2S}	&	$^1B_1$  		&6.141	&6.098	&6.110	&6.098	&6.102	&6.103	&6.103	&6.103(1)	&6.294	&6.240	&6.237	&6.238	&6.238	&6.240(7)\\
				&	$^1A_2$  		&6.343	&6.293	&6.299	&6.286	&6.286	&6.286	&6.286	&6.286(0)	&6.246	&6.192	&6.185	&6.181	&6.181	&6.181(6) \\
	\ce{H2O}	&	$^1B_1$  		&7.447	&7.511	&7.510	&7.497	&7.531	&7.528	&7.532	&7.533(0)	&7.597	&7.605	&7.591	&7.623	&7.620	&7.626(3)\\
				&	$^1A_2$  		&9.213	&9.293	&9.291	&9.279	&9.317	&9.313	&9.318	&9.318(0)	&9.361	&9.382	&9.368	&9.405	&9.400	&9.407(7)\\
				&	$^1A_1$  		&9.861	&9.921	&9.919	&9.903	&9.940	&9.937	&9.941	&9.941(0)	&9.957	&9.966	&9.949	&9.986	&9.981	&9.986(2)\\
	\hline
	MAE			&					&	\\
	\end{tabular}
	\end{ruledtabular}
\end{table*}
\end{squeezetable}














%%%%%%%%%%%%%%%%%%%%%%%%%%%%%%%%
\section{Conclusion}
%%%%%%%%%%%%%%%%%%%%%%%%%%%%%%%%

\titou{Here comes the conclusion.}

%%%%%%%%%%%%%%%%%%%%%%%%%%%%%%%%
\begin{acknowledgements}
This work was performed using HPC resources from CALMIP (Toulouse) under allocation 2021-18005.
PFL has received funding from the European Research Council (ERC) under the European Union's Horizon 2020 research and innovation programme (Grant agreement No.~863481).
\end{acknowledgements}
%%%%%%%%%%%%%%%%%%%%%%%%%%%%%%%%

%%%%%%%%%%%%%%%%%%%%%%%%%%%%%%%%
\section*{Data availability statement}
%%%%%%%%%%%%%%%%%%%%%%%%%%%%%%%%
The data that support the findings of this study are openly available in Zenodo at \href{http://doi.org/XX.XXXX/zenodo.XXXXXXX}{http://doi.org/XX.XXXX/zenodo.XXXXXXX}.

%%%%%%%%%%%%%%%%%%%%%%%%%%%%%%%%
\bibliography{CC4}
%%%%%%%%%%%%%%%%%%%%%%%%%%%%%%%%

\end{document}
